%%%%%%%%%%%%%%%%%%%%%%%%%%%%%
% Standard header for working papers
%
% WPHeader.tex
%
%%%%%%%%%%%%%%%%%%%%%%%%%%%%%

\documentclass[12pt]{article}

%%%%%%%%%%%%%%%%%%%%
%% Include general header where common packages are defined
%%%%%%%%%%%%%%%%%%%%



%%%%%%%%%%%%%%%%%%%%%%%%%%
%% TEMPLATES
%%%%%%%%%%%%%%%%%%%%%%%%%%


% Simple Tabular

%\begin{tabular}{ |c|c|c| } 
% \hline
% cell1 & cell2 & cell3 \\ 
% cell4 & cell5 & cell6 \\ 
% cell7 & cell8 & cell9 \\ 
% \hline
%\end{tabular}





%%%%%%%%%%%%%%%%%%%%%%%%%%
%% Packages
%%%%%%%%%%%%%%%%%%%%%%%%%%

\usepackage{times}

% encoding 
\usepackage[utf8]{inputenc}
\usepackage[T1]{fontenc}


% general packages without options
\usepackage{amsmath,amssymb,amsthm,bbm}

% graphics
\usepackage{graphicx,transparent,eso-pic}

% text formatting
\usepackage[document]{ragged2e}
\usepackage{pagecolor,color}
%\usepackage{ulem}
\usepackage{soul}







%%%%%%%%%%%%%%%%%%%%%%%%%%
%% Maths environment
%%%%%%%%%%%%%%%%%%%%%%%%%%

%\newtheorem{theorem}{Theorem}[section]
%\newtheorem{lemma}[theorem]{Lemma}
%\newtheorem{proposition}[theorem]{Proposition}
%\newtheorem{corollary}[theorem]{Corollary}

%\newenvironment{proof}[1][Proof]{\begin{trivlist}
%\item[\hskip \labelsep {\bfseries #1}]}{\end{trivlist}}
%\newenvironment{definition}[1][Definition]{\begin{trivlist}
%\item[\hskip \labelsep {\bfseries #1}]}{\end{trivlist}}
%\newenvironment{example}[1][Example]{\begin{trivlist}
%\item[\hskip \labelsep {\bfseries #1}]}{\end{trivlist}}
%\newenvironment{remark}[1][Remark]{\begin{trivlist}
%\item[\hskip \labelsep {\bfseries #1}]}{\end{trivlist}}

%\newcommand{\qed}{\nobreak \ifvmode \relax \else
%      \ifdim\lastskip<1.5em \hskip-\lastskip
%      \hskip1.5em plus0em minus0.5em \fi \nobreak
%      \vrule height0.75em width0.5em depth0.25em\fi}



%%%%%%%%%%%%%%%%%%%%
%% Idem general commands
%%%%%%%%%%%%%%%%%%%%
%% Commands

\newcommand{\noun}[1]{\textsc{#1}}


%% Math

% Operators
\DeclareMathOperator{\Cov}{Cov}
\DeclareMathOperator{\Var}{Var}
\DeclareMathOperator{\E}{\mathbb{E}}
\DeclareMathOperator{\Proba}{\mathbb{P}}

\newcommand{\Covb}[2]{\ensuremath{\Cov\!\left[#1,#2\right]}}
\newcommand{\Eb}[1]{\ensuremath{\E\!\left[#1\right]}}
\newcommand{\Pb}[1]{\ensuremath{\Proba\!\left[#1\right]}}
\newcommand{\Varb}[1]{\ensuremath{\Var\!\left[#1\right]}}

% norm
\newcommand{\norm}[1]{\left\lVert #1 \right\rVert}



% argmin
\DeclareMathOperator*{\argmin}{\arg\!\min}


% amsthm environments
\newtheorem{definition}{Definition}
\newtheorem{proposition}{Proposition}
\newtheorem{assumption}{Assumption}

%% graphics

% renew graphics command for relative path providment only ?
%\renewcommand{\includegraphics[]{}}





\renewcommand{\abstractname}{}




% geometry
\usepackage[margin=1.5cm]{geometry}

% layout : use fancyhdr package
\usepackage{fancyhdr}
\pagestyle{fancy}

\makeatletter

\renewcommand{\headrulewidth}{0.4pt}
\renewcommand{\footrulewidth}{0.4pt}
\fancyhead[RO,RE]{\textit{Théo Quant 2017}}
\fancyhead[LO,LE]{Co-construire Modèles et Théories en TQG}
\fancyfoot[RO,RE] {\thepage}
\fancyfoot[LO,LE] {\noun{J. Raimbault}}
\fancyfoot[CO,CE] {}

\makeatother


%%%%%%%%%%%%%%%%%%%%%
%% Begin doc
%%%%%%%%%%%%%%%%%%%%%

\begin{document}









\title{\vspace{-1.5cm}Towards multi-scalar models for the co-evolution of transportation networks and territories\\\medskip
\textit{Théo Quant 2019 - Communication proposal}\\
}
\author{\noun{Juste Raimbault}$^{1,2}$\\
$^1$ UPS CNRS 3611 ISC-PIF\\
$^2$ UMR CNRS 8504 Géographie-cités
}
%\date{Novembre 2016}
\date{}

\maketitle

\justify


\begin{abstract}
\end{abstract}

\pagenumbering{gobble}

\vspace{-1cm}

\textbf{Keywords: Transportation networks; Territories; Co-evolution; Modeling; Multi-scale}

\bigskip

The intricate relations between transportation networks and territories, at multiples scales, has fed numerous open questions such as the potential existence of structuring effects of networks \cite{offner1993effets}. \cite{raimbault2018caracterisation} explored these interactions from the point of view of co-evolutive dynamics, aiming at modeling these. In particular, \cite{raimbault2018modeling} introduced a co-evolution model at the macroscopic scale for systems of cities with abstract networks, whereas \cite{raimbault2018urban} developed a morphogenesis model at the mesoscopic scale, capturing the co-evolution between road networks and the population density grid. The processes included in models depend naturally on the scale. This communication aims at showing the necessity of a multi-scalar approach for a more accurate account of these co-evolutive dynamics. We introduce first a physical implementation of transportation networks into the macroscopic model, with a description of networks with a 1km spatial resolution, whereas the typical range of application of the model is around 1000km. Systematic explorations of the model on synthetic data using the OpenMole software show that even with the same processes for network growth, qualitative behavior fundamentally differ. In particular, co-evolutive dynamics are more difficult to characterize in the hybrid scale models, recalling results obtained by \cite{raimbault2018caracterisation} with the SimpopNet model \cite{schmitt2014modelisation} which has a similar structure. Calibration on the French system of cities with the objective of population and distance matrices accurateness gives mitigated results in comparison to the original model, but however witnesses fit improvement on several temporal calibration windows. These results suggest that the hybrid model would be closer to the actual complexity of these dynamics, as it was shown by \cite{raimbault2018modeling} that co-evolution is indeed difficult to characterize empirically for the French system of cities with railway network data. We finally discuss from a theoretical point of view what would be the advantages of a multi-scale coupling of the macroscopic and the mesoscopic model (including for example a more operational character, more accurate conditioning of local dynamics by exogenous parameters determined at the macroscopic scale, a better grasp on spatio-temporal non-stationarity) and present different modeling alternatives that could be followed to achieve this coupling.





%%%%%%%%%%%%%%%%%%%%
%% Biblio
%%%%%%%%%%%%%%%%%%%%

\footnotesize

\bibliographystyle{apalike}
\bibliography{biblio}


\end{document}
